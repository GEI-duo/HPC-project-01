% \usepackage{tabularray}
\begin{table}
    \centering
    \begin{tblr}{
      row{1} = {c},
      column{3} = {c},
      column{4} = {c},
      column{5} = {c},
      cell{1}{2} = {c=5}{},
      cell{2}{1} = {c},
      cell{2}{2} = {c},
      cell{3}{1} = {c},
      cell{3}{2} = {c},
      cell{4}{1} = {c},
      cell{4}{2} = {c},
      cell{5}{1} = {c},
      cell{5}{2} = {c},
      hlines,
      vlines,
    }
    \textbf{\textit{100 STEPS}}   & \textbf{\textit{Execution}} &                           &                           &                           &                           \\
    \textbf{\textit{Matrix Size}} & \textbf{\textit{SERIAL}}    & \textbf{\textit{OMP T-1}} & \textbf{\textit{OMP T-2}} & \textbf{\textit{OMP T-4}} & \textbf{\textit{OMP T-8}} \\
    \textbf{\textit{100x100}}     & \textit{0.01}               & \textit{0.023784}         & \textit{0.014906}         & \textit{0.010339}         & \textit{0.015886}         \\
    \textbf{\textit{1000x1000}}   & \textit{1.34}               & \textit{1.667607}         & \textit{1.007961}         & \textit{0.621634}         & \textit{0.670591}         \\
    \textbf{\textit{2000x2000}}   & \textit{5.33}               & \textit{6.344592}         & \textit{3.766311}         & \textit{2.432052}         & \textit{2.452237}         
    \end{tblr}
    \caption{100 steps execution time results}
  \end{table}