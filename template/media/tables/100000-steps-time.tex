% \usepackage[normalem]{ulem}
% \usepackage{tabularray}
\begin{table}
    \centering
    \begin{tblr}{
      cells = {c},
      cell{1}{2} = {c=5}{},
      hlines,
      vlines,
    }
    \textbf{\textit{100000 STEPS}} & \textbf{\textit{Execution}} &                           &                           &                           &                           \\
    \textbf{\textit{Matrix Size}}  & \textbf{\textit{SERIAL}}    & \textbf{\textit{OMP T-1}} & \textbf{\textit{OMP T-2}} & \textbf{\textit{OMP T-4}} & \textbf{\textit{OMP T-8}} \\
    \textbf{\textit{100x100}}      & 10.18                       & 12.819621                 & 6.75271                   & 3.657432                  & 7.254863                  \\
    \textbf{\textit{1000x1000}}    & 1072.98                     & 1331.956969               & 687.403504                & 358.492834                & 377.120077                \\
    \textbf{\textit{2000x2000}}    & 4310.41                     & 5374.222172               & 2820.720803               & 1440.691108               & 1493.628449               
    \end{tblr}
    \caption{100000 steps execution time results}
\end{table}